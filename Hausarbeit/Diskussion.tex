\section{Diskussion} % (fold)
    \label{sec:diskussion}
    Betrachtet man die \nameref{sec:ergebnisse}, zeichnet sich das salzige Medium als eindeutig signifikant beeinflussenden Faktor ab. Die Hypothesentests in \autoref{tab:t_tests} liefern eine 100\%-ige Bestätigung dessen, und auch in \autoref{fig:t_tests} ist dies abzulesen, da alle Messwerte außerhalb des rechtsseitigen 97,5\%-Quantils liegen.

    Das saure Medium ist etwas weniger eindeutig. Zwar liegen die Werte im Mittel immer über denen des normalen Mediums (\autoref{fig:scat_plot}), weisen aber nur zu etwas unter 30\% einen signifikanten Unterschied zu diesen auf (\autoref{tab:t_tests}). Dieses Ergebnis ist auch aus \autoref{fig:box_plot} ableitbar. Im Verlauf ähneln sich die Messungen stark, nur dass sie beim sauren Medium weniger streuen. Möglicherweise kommt diese Streuung dadurch zustande, dass die Säure die Samenhülle angreift und somit das Durchbrechen des Keims erleichtert. Vielleicht sorgt ebendieser `Vorteil' auch für den kleinen, unaufholbaren Vorsprung zum normalen Medium, sodass sich das Wachstum beider Medien eigentlich gleich verhält. Anhand dieser Werte scheint der \hyperref[sec:einleitung]{anfänglich} beschriebene negative Einfluss der Säure auf die Nährstoffaufnahme nicht zu greifen, entweder durch Resistenz oder weil der Samen genug Nährstoffe für das gesamte Wachstum enthält.

    Allerdings scheinen die Referenzdaten von diesem Resultat abzuweichen. In \autoref{fig:scat_plot_cmp} sind deren Verläufe vergleichend dargestellt. Die Wachstumsentwicklung des salzigen Mediums ähnelt der der Referenz stark, sodass dieses Ergebnis nicht betroffen ist. Zwar sind die Pflanzen bei Normalbedingungen dort weniger hoch gewachsen, unterscheiden sich aber dennoch stark vom salzigen Verlauf.

    Bei der sauren Probe hingegen liegen alle Werte deutlich unter denen der Normalen. Zwischen Tag 12 und 13 ist zwar ein plötzlicher Abfall sichtbar, aber vermutlich wurden hier teilweise NA-Werte als 0 ausgewertet. Aber unabhängig davon stellt sich hier die Frage, wo dieser große Unterschied zu den selbst erhobenen Daten herrührt.

    Eine mögliche Erklärung hierfür ist, dass die Referenzdaten unter schlechteren Umweltbedingungen erhoben wurden, sodass die Pflanzen in ihrer Entwicklung fundamental eingeschränkt waren. Da diese Daten allerdings aus fünf voneinander unabhängigen Durchführungen stammen, ist diese Annahme nicht sehr wahrscheinlich, aber dennoch nicht unmöglich.

    Deutlich naheliegender ist ein Fehler in der Erzeugung der eigenen Daten. Die Materialienliste in \autoref{tab:materials} betrachtend, erschließen sich mir zwei Theorien:
    \begin{enumerate}[1.]
        \item \label{th:1} Die Lagerung des Essigs hat dessen Wirkung beeinflusst
        \item \label{th:2} Die verschließbaren Gläser wurden nicht gründlich genug gereinigt
    \end{enumerate} 
    \hyperref[th:1]{Die erste Theorie} erklärt zwar nicht den Unterschied des Wachstums zum normalen Medium, kann aber dennoch einen relevanten Einfluss auf das Experiment gehabt haben. Die benötigte Menge wurde in ein eigenes Gefäß abgefüllt und dann bis zum Tag 0 bei Zimmertemperatur dunkel gelagert. Während der Durchführung wurde das saure Medium bei Tageslicht gelagert. Möglicherweise hatten Lichtverhältnisse oder Temperatur bei der Lagerung keine neutrale Rolle gespielt.

    \hyperref[th:2]{Die zweite Theorie} kann beide Unterschiede erklären. Die Gläser wurden vorher für die Aufbewahrung von Lebensmitteln verwendet. Wenn die Reinigung davon nicht ausreichend gewesen ist und dementsprechen noch Nährstoffrückstände an der Innenwand erhalten wurden, könnten diese einen düngenden Effekt auf die Kresse gehabt haben. Das begründet das erhöhte Wachstum. Dass die Wuchshöhe beim sauren Medium dem des Normalen ähnelt, hängt vermutlich mit der maximalen Wuchshöhe von Kresse zusammen. Angenommen, die genutzte Gartenkresse könnte unbeschränkt wachsen, wären die Verläufe meiner Messwerte in \autoref{fig:scat_plot_cmp} vermutlich proportional zu der Referenz. In \autoref{fig:acidic_after} ist auf jeden Fall ein Einfluss der Säure zu erkennen, da einige Blattspitzen leicht gelb sind, sofern das kein Zufall ist.

    Die \hyperref[hypothese]{Ausgangsfrage}, ob es neben tiefen Temperaturen noch andere wachstumsbeeinflussende Extremfaktoren gibt, wird durch diese Resultate teilweise bestätigt, da Gartenkresse zumindest in sehr salziger Umgebung stark eingeschränkt zu wachsen scheint.

    Im Hinblick auf ein saures Umfeld sollte das Experiment nochmal wiederholt werden, allerdings mit sterilen Gefäßen für die Lagerung des Mediums. Zudem schlage ich vor, die sauer bestellten Kressesamen in drei Gruppen aufzuteilen. Die erste wird mit Medium gewässert, dass bei Tageslicht gelagert wird, die zweite mit welchem bei dunkler Lagerung, und die dritte mit welchem bei kühler, dunkler Lagerung. Gibt es dann Unterschiede zwischen Hell und Dunkel bei Normaltemperatur oder zwischen Dunkel bei normaler und Dunkel bei kühler Temperatur, wird dadurch \hyperref[th:1]{Theorie 1} geprüft.

    Zudem besteht noch die Möglichkeit, das Wachstum hinblicklich anderer Einflüsse zu beobachten. So könnte man das Verhalten bei unterschiedlichen Lichtverhältnissen auswerten oder bei Überschuss oder Mangel bestimmter Nährstoffe. Auch ist die Konsistenz des Wuchsbodens ein denkbarer Faktor.

    Es besteht also noch viel Potenzial für weitergehende Experimente.
% section diskussion (end)