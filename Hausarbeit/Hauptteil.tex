\section{Material und Methoden} % (fold)
\label{sec:material_und_methoden}
    Es werden drei Versuchsmedien erstellt, um das Wachstum differenziert betrachten zu können. Dabei wird eines mit normalem Leitungswasser gewässert, eines mit Salzwasser und eines mit angesäuertem Wasser. Der Salzgehalt soll bei $0,1 \frac{mol}{l}$ liegen. Dementsprechend werden dazu 5,85g NaCl in 100ml Leitungswasser gelöst und anschließend zu einem Liter aufgefüllt (Werte siehe Berechnung~\ref{equ:nacl}).
    \begin{equation}\label{equ:nacl}
        \begin{split}  
        m(NaCl) &= 0,1\frac{mol}{l} * V(H_2O) * M(NaCl)\\
        &= 0,1\frac{mol}{l} * 1l * 58,5\frac{g}{mol}\\
        & = 5,85g
        \end{split}
    \end{equation}
    Das saure Medium soll einen pH-Wert von 5 haben. Zur Verfügung steht Essigessenz mit dem pH-Wert 2,65, sodass von dieser 4,5ml zu einem Liter aufgefüllt werden (siehe Berechnung~\ref{equ:essig}).
    \begin{equation}
        \label{equ:essig}
        \begin{split}
            pH(Essig) & = 2,65 \\
            \Rightarrow c(H^+)_{alt} &=10^{-2,65}\\
            c(H^+)_{neu}&=10^{-5}=10^{-2,65} * x * l\\
            x&=0,0045l
            % &&= 5,85
        \end{split}
    \end{equation}
    \begin{table}[h]
        \caption{Materialien}
        \label{tab:materials}
        \csvreader[tabular=rrrr,
            respect percent,
            table head=\toprule\centerIt{\textbf{Material}} & \centerIt{\textbf{Menge}} & \centerIt{\textbf{Zweck}} & \centerIt{\textbf{Zusatzinfo}}\\\midrule,
            /csv/separator=semicolon,
            head to column names,
            late after last line=\\\bottomrule]
            {../Materials.csv}{}
            {\Material&\Menge&\Zweck&\Zusatzinfo}
    \end{table}
    \subsection{Durchführung} % (fold)
    \label{sub:durchführung}
    
    % subsection durchführung (end)
% section material_und_methoden (end)

\newpage
\section{Ergebnisse} % (fold)
    \label{sec:ergebnisse}
    Text

    \begin{table}[h]
        \centering
        \caption{t-Test-Ergebnisse}
        \label{tab:t_test}
        \csvreader[tabular=rrr,
          table head=\toprule\centerIt{\textbf{Nullhypothese}} & \centerIt{\textbf{Sauer}} & \centerIt{\textbf{Salzig}}\\\midrule,
          /csv/separator=semicolon,
          head to column names,
          late after line=\\,
          late after last line=\\\bottomrule
          ]%
          {../Results/t_tests.csv}{H0=\H}%
        {\H & \pH & \NaCl}
    \end{table}
    Wie in Tabelle~\ref{tab:t_test} zu sehen ist, \dots Die Quantile sind bla und bla\ \cite[vgl.][]{web:t-values} und bla\ \cite[vgl.][]{web:Gartenratgeber}
% section ergebnisse (end)
