\section{Material und Methoden} % (fold)
\label{sec:material_und_methoden}
    Es werden drei Versuchsmedien erstellt, um das Wachstum differenziert betrachten zu können. Dabei wird eines mit normalem Leitungswasser gewässert, eines mit Salzwasser und eines mit angesäuertem Wasser. Der Salzgehalt soll bei $0,1 \frac{mol}{l}$ liegen. Dementsprechend wird dazu ein Liter Leitungswasser mit 5,85g NaCl gemischt, wie der Berechnung~\ref{equ:nacl} zu entnehmen ist.
    \begin{equation}\label{equ:nacl}
        m(NaCl) = 0,1\frac{mol}{l} * V(H_2O) * M(NaCl)
        = 0,1\frac{mol}{l} * 1l * 58,5\frac{g}{mol}
        % \begin{aligned}
            
        % = 5,85g
        % \end{aligned}
    \end{equation}
    % \begin{equation}
    %     \label{equ:essig}
    %         pH(Essig) = 2,65
    %         \Rightarrow c(H^+)=
    %         = 5,85g
    % \end{equation}
    % \begin{table}
    %     \centering
    %     \caption{Materialien}
    %     \label{tab:materials}
    %     \begin{tabular}{rrrr}
    %         \toprule
    %         \multicolumn{1}{c}{\bfseries Material} & \multicolumn{1}{c}{\bfseries Menge} & \multicolumn{1}{c}{\bfseries Zweck} & \multicolumn{1}{c}{\bfseries Zusatzinfo}\\ 
    %         \midrule
    %         \csvreader[respect percent,
    %           /csv/separator=semicolon,
    %           head to column names,
    %           late after line=\\,
    %           column count=4,
    %           ]%
    %           {../Materials.csv}{Material=\Material, Menge=\Menge, Zweck=\Zweck, Zusatzinfo=\Zusatzinfo}%
    %         {\Material & \Menge & \Zweck & \Zusatzinfo}
    %         \bottomrule
    %     \end{tabular}
    % \end{table}
    \subsection{Durchführung} % (fold)
    \label{sub:durchführung}
    
    % subsection durchführung (end)
% section material_und_methoden (end)

\section{Ergebnisse} % (fold)
    \label{sec:ergebnisse}

    \begin{table}
        \centering
        \caption{t-Test-Ergebnisse}
        \label{tab:t_test}
        \begin{tabular}{rrr}
            \toprule
            \multicolumn{1}{c}{\bfseries Nullhypothese} & \multicolumn{1}{c}{\bfseries Sauer} & \multicolumn{1}{c}{\bfseries Salzig} \\ 
            \midrule
            \csvreader[
              /csv/separator=semicolon,
              head to column names,
              late after line=\\,
              column count=3,
              ]%
              {../Results/t_tests.csv}{H0=\H, pH=\pH, NaCl=\NaCl}%
            {\H & \pH & \NaCl}
            \bottomrule
        \end{tabular}
    \end{table}
    Wie in Tabelle~\ref{tab:t_test} zu sehen ist, \dots
    \cite{aristotle:physics}. Die Quantile sind bla und bla\ \cite[vgl.][]{web:t-values} und bla\ \cite[vgl.][]{web:Gartenratgeber}
% section ergebnisse (end)
