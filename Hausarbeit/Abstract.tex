\phantomsection  % important for correct reference in toc
\label{sec:abstract}
\addcontentsline{toc}{section}{Abstract}
    \begin{center}
      \textbf{Zusammenfassung}
    \end{center}
    Diese Arbeit beschäftigt sich mit der Frage, wie sich das Wachstumsverhalten von Gartenkresse unter extremen Bedingungen verändert, hierbei in Betrachtung der Salzkonzentration und der Säure des Gießwassers. Die Fragestellung ist dadurch entstanden, dass Gartenkresse scheinbar ziemlich anspruchslos weltweit verbreitet ist und gedeiht. Um das Problem anzugehen, wurde ein Experiment entwickelt und mehrfach durchgeführt, in dem über 14 Tage die Wuchshöhen dreier Versuchsgruppen von Gartenkresse gemessen werden. Eine der Gruppen wird dabei mit normalem Leitungswasser gegossen und dient somit als Nullprobe zur Referenz. Die zweite Gruppe wird mit salzigem und die Dritte mit saurem Medium bewässert. Im Anschluss werden die Ergebnisse ausgewertet, wonach Salz definitiv einen hemmenden Einfluss auf das Wachstum zu haben scheint. Für das saure Umfeld müssen noch Nachuntersuchungen erfolgen, da die Messwerte der hier betrachteten Durchführung sehr von denen der anderen abweichen.
\vspace{4em}
    \begin{otherlanguage}{english}
        \begin{center}
          \textbf{Abstract}
        \end{center}
        This thesis deals with the question of how the growth behavior of garden cress changes under extreme conditions, considering the salt concentration and the acidity of the irrigation water. The question arose from the fact that garden cress seems to grow undemanding around the globe. To face the problem, an experiment was designed and carried out several times, in which the growth heights of three experimental groups of garden cress were measured over 14 days. One of the groups is watered with normal tap water and serves as a blank sample for reference. The second group is watered with salty and the third with acidic medium. Afterwards, the results are evaluated, according to which salt definitely seems to have an inhibiting influence on the growth. For the acidic environment, further investigations are necessary, as the measured values of this run differ greatly from those of the others.
    \end{otherlanguage}