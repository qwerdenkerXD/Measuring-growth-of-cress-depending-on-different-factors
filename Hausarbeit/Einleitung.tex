\section{Einleitung} % (fold)
    \label{sec:einleitung}
    Kressen (\textit{Lepidium}) sind krautige oder strauchartige Pflanzen der Familie der Kreuzblütergewächse und werden vielseitig in der Küche verwendet, da sie vitaminhaltig sind und sich gut zum Würzen von Speisen eignen\ \cite[vgl.][]{web:meinschoenergarten}. Es gibt sehr viele verschiedene Arten von Kresse, doch am weitesten verbreitet ist die Gartenkresse (\textit{Lepidium sativum}), um die es im Folgenden gehen wird. Dass sie vielerorts vorkommt, hängt damit zusammen, dass diese Art von Kresse hinsichtlich ihres Wuchsumfelds sehr anspruchslos ist und daher nahezu überall wächst, sofern Plusgrade herrschen\ \cite[vgl.][]{web:meinschoenergarten}. Letzteres wirft die Frage auf, ob es neben sehr tiefen Temperaturen noch andere Extrembedingungen gibt, die sich negativ auf das Wachstum auswirken.\par
    Ein solcher Faktor ist bei vielen Pflanzen der Salzgehalt im Boden, denn ``hohe Salzkonzentrationen in den Böden verursachen bei Pflanzen [\dots] sogenannten Salzstress. Dieser hemmt das Wachstum [\dots] und kann in Pflanzen gar zum Tod führen''\ \cite{web:salzstress}. Auch der pH-Wert ist beim Pflanzenwachstum von Bedeutung, da die Nährstoffaufnahme bei sonderbar sauren oder alkalischen Bedingungen beeinträchtigt wird\ \cite[vgl.][]{web:phwert} und zudem auch die Funktionalität der pflanzeneigenen Enzyme, die meist bei neutralem Umfeld optimal katalysieren.\par
    Diesbezüglich soll ein Experiment stattfinden, das das Wachstum vom Samen unter solchen Bedingungen für zwei Wochen beobachtet. Es ist anzunehmen, dass erstgenannter Faktor mehr Einfluss auf das Wachstum haben wird als der pH-Wert, da Letzterer lediglich die Nährstoffaufnahme beeinträchtigt, was nicht notwendigerweise ein Problem darstellt, wenn die entsprechenden Nährstoffe aus den Samen gezogen werden können, während der Salzgehalt durch den kausierten osmotischen Stress viel tiefer in den Wachstumsprozess eingreift.
% section einleitung (end)